\section{Cloud Grading Architecture With OpenEdX}



a.	queue system and openEdX; can connect to other stuff as well


\begin{figure}
  \caption{\label{fig:arch}
  Students submit each assignment by posting a Web form to a designated
  URI.  etc}

\end{figure}

As Figure~\ref{fig:arch} shows, 
our autograders follow a producer-consumer architecture that we
originally developed to work with Coursera's (cite) infrastructure and
subsequently modified to work with the open-source edX framework (cite).
Since each type of autograder relies on many libraries,
tools, support files, and so on, we encapsulate each in its own VM image.
At VM startup, the autograder script automatically runs from
\texttt{/etc/init.d} and examines a deploy-time environment variable
to get the base URI of the queue name from which to draw assignments.  
This URI is an endpoint that accepts three RESTful request types:

\begin{enumerate}
\item Check queue length (number of assignments waiting to be graded).
  The implementation of the queue may be distributed, but the endpoint
  presents a single logical queue.
\item Retrieve a submission.  The return value is a submission along
  with an opaque assignment-ID and an opaque student-ID.
\item Post a grade.  The request includes the assignment-ID, student-ID,
  grade (a short string or numeric value), and freeform text feedback.
\end{enumerate}

If the queue is empty, the autograder sleeps for a few minutes and checks
again; this machinery will soon be extracted into a separate manager
that automatically manages the number of active autograder instances
based on the queue length.

Once an autograder retrieves a submission to grade, it must post a grade
within a configurable timeout period (typically a few minutes).
If the timeout expires, the queue server is responsible for re-queueing
that assignment for regrading.  (TBD: at some point if it's queued too
many times we should time it out, as it may be pathological)

We deploy the autograders on Amazon EC2.  We have many reasons for
deploying on the cloud vs. the client (for example, autograders that run on
the student's own computer, as in (cite AutoGradeMe, some of the AI/ML
ones, others),  for using a stateless client-server
architecture rather

Scalability to MOOC size using an elastic cloud.  In our design,
  the autograders 
  themselves are stateless; the only state is in the queue manager,
  which must keep track of assignments whose autograder times out.  The
  stateless design allows us to exploit the public cloud's elasticity to
  scale the number of 
  autograder instances rapidly and allow students to get quick
  feedback.  Even our most sophisticated autograders take less than one
  machine-minute per assignment; at less than 10 cents per machine-hour,
  MOOC-scale 
  autograding is cost-effective.

b.	autograder, rubric file

c.	CI workflow for ensuring autograders running

d.	Multiple layers of security: watchdog timers, sandboxed interpreter, threads. Hollingsworth 1960 observed that it was possible for students to submit programs that deliberately damage the autograder.

 Trustworthiness: because we control the VM image in which the
  autograder runs and the grades are recorded to a central queue server
  which we also control, students cannot tamper with the autograder's
  results or functioning.

 Sandboxing: while our RSpec-based autograder performs much of its
  own sandboxing, even if student code escapes the sandbox we can shut
  down and undeploy the VM.  

  TBD Watchdog timers that do this if miss several heartbeats.



e.	Policy: resubmission, plagiarism detection, formative vs summative, avoding "autograder-driven development"



