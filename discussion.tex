\section{Discussion}



a.	Value of tools ecosystem in Rails world. Majority of new autograders in 2010 survey were either for Java or language-neutral with output checking only. We had good reasons for using Ruby and still do. Student feedback in campus course bears this out.


When we set about revising the Software Engineering course in which
these autograders are used, our choice of Ruby and Rails was not without
controversy.  Some students complained about having to learn another
language (the three lower-division courses that form the gateway into
the major are taught in Python, Java, and C), despite our assurances
that they would need this skill  as professional developers, learning a
new language every three or four years.  Our view was, and continues to
be, that the superior tools for testing, code analysis and code grooming
available in Ruby resulted in higher productivity and student engagement
that justified this learning curve.  What we did not anticipate was that
we ourselves would ultimately turn to these same tools to facilitate
automated grading.  Of the three autograders we have described, only the
first, using RSpec, is Ruby-specific.



b.	Getting unnecessary stuff out of the way: avoid wasting student time with admin, setup, etc.  

1.	VM provides courseware; maybe soon C9.  

2.	Heroku provides deployment.  

3.	OpenEdX presents the HW explanations, videos, selfcheck questions along the way using RuQL (cite), inspired by ideas like OKgrader (cite)

c.	Challenge: tuning rubrics. Use campus course to debug.

d.	Challenge: stability. By and large edX works.  Good separation
of concerns between LMS and autograder authors.

e.	Challenge: test suite quality. THis si a general problem in SWE. Example: if multiple tests are effectively redundant, student scoring is positively distorted if all pass and negatively distorted if all fail.



