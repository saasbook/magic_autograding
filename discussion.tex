\section{Discussion}


\subsection{The Value of Great Tools}


When we set about revising the Software Engineering course in which
these autograders are used, our choice of Ruby and Rails was not without
controversy.  Some students complained about having to learn another
language (the three lower-division courses that form the gateway into
the major are taught in Python, Java, and C), despite our assurances
that they would need this skill  as professional developers, learning a
new language every three or four years.  Our view was, and continues to
be, that the superior tools for testing, code analysis and code grooming
available in Ruby resulted in higher productivity and student engagement
that justified this learning curve.  What we did not anticipate was that
we ourselves would ultimately turn to these same tools to facilitate
automated grading.  Of the three autograders we have described, only the
RSpecGrader is Ruby-specific.

\subsection{Zero-Configuration Autograders}

Both  surveys of autograders ask why existing autograders aren't reused more.
We believe one reason is the configuration required for teachers to deploy them and
students to submit work to them.  Since we
faced and surmounted this problem in deploying our ``autograders as a
service'' with OpenEdX, we can make them easy for others to use.

We already have several instructors running SPOCs based on our
materials~\cite{moocs-spocs-TR} using OpenEdX.  They not only use our
autograders but can create new assignments that take advantage of them.

or create another adapter than XQueue - refer to figure - we're now
doing it for Octobear

\subsection{Challenge: Tuning Rubrics}

tuning rubrics. Use campus course to debug. Use CI to
verify we haven't broken stuff.

test suite quality: THis si a general problem in SWE. Example: if multiple tests are effectively redundant, student scoring is positively distorted if all pass and negatively distorted if all fail.


\subsection{Challenge: Avoiding ``Autograder-Driven Development''}

\subsection{Challenge: Stability}

By and large edX works.  Good separation
of concerns between LMS and autograder authors.




