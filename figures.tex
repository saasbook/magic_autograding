
%%%% figures
\clearpage
\section*{Appendix: Code Examples}

\begin{figure} \centering
  \lstinputlisting[numbers=left,tabsize=1,basicstyle=\scriptsize\ttfamily]{figs/rspec_grader_rubric.rb}
  \caption{\label{fig:rspec_grader_rubric} In an RSpecGrader rubric,
    some test cases are ``sanity checks'' without which the assignment
    isn't even graded (lines 2--9) while others contribute points to the
    student's score.
  Ruby's dynamic language features allow us to
    easily check, for example, that
a student's implementation of a ``sum all the numbers'' method does not
simply call a built-in library method (line 7).
  }
\end{figure}


\begin{figure}
 \centering
  \lstinputlisting[language=Ruby,numbers=left]{figs/mechanize_grader_example.rb}
  \caption{\label{fig:mechanize_grader_example} 
This excerpt of three test cases from a MechanizeGrader rubric runs
against a student's 
deployed full-stack application.}
\end{figure}

\begin{figure}
  \centering
    \lstinputlisting{figs/cucumber_example.feature}
    \lstinputlisting[language=Ruby]{figs/cucumber_step_def_example.rb}%
  \caption{\label{fig:cucumber} Cucumber accepts integration tests 
    written in stylized prose (top) and uses regular expressions to map each
    step to a \emph{step definition} (bottom) that sets up preconditions, exercises the app,
    or checks postconditions.  Step definitions 
    can stimulate a full-stack GUI-based web application in various
    ways, including remote-controlling a real browser with Webdriver
    (formerly Selenium) or using the Ruby Mechanize library to interact
    with a remote site.  Our code blocks are in Ruby, but the Cucumber framework
itself is polyglot.} 
\end{figure}

 
