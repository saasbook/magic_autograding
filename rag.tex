\section{\emph{rag}, a Ruby Autograder for ESaaS}

Having chosen Ruby and Rails for their excellent testing and
code-grooming tools, our approach was to repurpose those same tools into
autograders that would give finer-grained feedback than human graders
using more detailed tests, and would be easier to repurpose than
those built for other languages.

\begin{figure}
  \centering
  \includegraphics[width=0.7\textwidth]{figs/rag.pdf}
  {\small
\begin{tabular}{|p{0.13\textwidth}|p{0.25\textwidth}|p{0.6\textwidth}|}
 \hline
 \textbf{Grader} & \textbf{Based on} & \textbf{Student Assignment Type} \\
 \hline
 RSpec Grader &
 unit testing &
 Student submits one or more class files for black-box and glass-box
 unit tests
 \\
\hline
 Mechanize Grader & 
 \raggedright remote integration testing &
 Black-box tests stimulate student app deployed on public cloud
 and parse/analyze output
 \\
\hline
 Feature Grader &
 \raggedright simplified mutation testing &
 Evaluates coverage and fragility of student-provided integration tests
 by inserting bugs into reference app 
 \\
 \hline
\end{tabular}
}

  \caption{\label{fig:grader_summary} Summary of the autograder
    engines based on our repurposing of excellent existing open-source
    tools and testing techniques.  Only the RSpecGrader is
    Ruby-specific.}
\end{figure}

\texttt{rag}\uf{github.com/saasbook/rag} is actually a collection of
three different autograding ``engines'' based on open-source testing
tools, as Figure~\ref{fig:grader_summary} shows.
Each engine takes as input a student-submitted work product and one or
more rubric files whose content depends on the grader
engine\footnote{Currently the rubric files must be present in the local
filesystem of the autograder VM, but refactoring is in progress to
allow these files to be securely loaded on-demand from a remote host
so that currently-running autograder VMs do not have to be modified
when an assignment is added or changed.}, and grades the work
according to the rubric.
The first of these (Figure~\ref{fig:grader_summary}, left) is
\textbf{RSpecGrader}, based on RSpec, an XUnit-like TDD framework that
exploits Ruby's flexible syntax to embed a highly readable unit-testing
DSL in Ruby.
The instructor annotates specific tests within an assignment with point
values (out of 100 total); RSpecGrader computes the total points
achieved and concatenates and formats the error/failure messages from
any failed tests, as Figure~\ref{fig:rspec_grader_rubric} shows.
RSpecGrader wraps the student code in a standard preamble and postamble
in which large sections of the standard library such as file I/O and
most system calls have been stubbed out, allowing us to handle
exceptions in RSpec itself as well as test failures.
RSpecGrader also runs in a separate timer-protected interpreter thread
to protect against infinite loops and pathologically slow student code.

A variant of RSpecGrader is \textbf{MechanizeGrader}  (Figure~\ref{fig:grader_summary}, center).
Surveys of recent
autograders~\cite{ihantola-2010-autograding-survey,douce-2005-autograding-survey}
mentioned as a ``future direction'' a grader that can assess full-stack
GUI applications.
MechanizeGrader does this using Capybara and
Mechanize\footnote{\url{jnicklas.github.io/capybara},
\url{rubygems.org/gem/mechanize}}.
Capybara implements a Ruby-embedded DSL for interacting with Web-based
applications by providing primitives that trigger actions on a web page
such as filling in form fields or clicking a button, and examining the
server's delivered results pages using XPath\uf{w3.org/TR/xpath20/}, as
Figure~\ref{fig:mechanize_grader_example} shows. 
Capybara is usually used as an in-process testing tool, but Mechanize
can trigger Capybara's actions against a remote application, allowing
black-box testing.
Students' ``submission'' to MechanizeGrader is therefore the URL to their
application deployed on Heroku's public 
cloud.

Finally, one of our assignments requires students to write integration-level
tests using Cucumber, which allows such tests to be formulated in
stylized plain text, as Figure~\ref{fig:cucumber} shows.
Our autograder for this
style of assignment is inspired by mutation testing, a technique invented
by George Ammann and Jeff 
Offutt~\cite{ammann-offutt-sw-testing} in which a
testing tool pseudo-randomly mutates the program under test to ensure
that some test fails as a result of these introduced errors.


\begin{figure}
  \begin{minipage}{0.45\textwidth}%
  \includegraphics[width=\textwidth]{figs/feature_grader.pdf}%
  \end{minipage}%
  \begin{minipage}{0.55\textwidth}%
  \lstinputlisting{figs/feature_grader_example.yml}%
  \end{minipage}
  \caption{\label{fig:featuregrader}%
FeatureGrader workflow and example YAML file.  In this case if Step1-1 passes,
Step1-3 will be run next.  Earlier steps must be less restrictive than
later steps (if the earlier step fails, there should be no way that a later one could pass).
\texttt{failures} are the two student-provided Cucumber scenarios that \emph{should fail} when
run because of mutations (bugs) inserted in the app.
}
\end{figure}


Specifically, \textbf{FeatureGrader}  (Figure~\ref{fig:grader_summary},
right)
operates by working with a
reference application designed so that its behavior can be modified by
manipulating certain environment variables.
Each student-created test   is first applied to the reference
application to ensure it passes when run against a known-good subject.
Next the FeatureGrader starts to mutate the reference application
according to a simple specification (Figure~\ref{fig:featuregrader}), 
introducing specific bugs and checking that some student-created test
does in fact fail in the expected manner in the presence of the
introduced bug.  
